%
%
\documentclass[a4paper]{article}
%
\usepackage{epsfig}
\usepackage{t1enc}
\usepackage{times}
% \usepackage[swedish]{babel}

\title{Route data V1.08}
\author{Wayfinder Systems}

% \frenchspacing

\begin{document}
\maketitle


\section{Introduction}

This document specifies the layout for route data in the Navigator Protocol.

\section{Needed Data}

The following data could be relevant for each route point:

\begin{itemize}

  \item Action to be performed at this point.
  \item Position of current point.
  \item Streetname to be displayed at this point.
  \item Crossing type.

  \item Distance to next point.
  \item Angle to next point.
  \item Speed limit used between this point and the next.

  \item Exit number (in a roundabout)
  \item Roadsign text to be displayed.

  \item Roadclass?
\end{itemize}

All data will need to fit into a structure that holds 6 {\tt uint16}
pieces of data.

\newpage

\section{Route data layout}

The route data starts with a start and ends with an end. If the route is
trunkated and protoVer >= 0x05 then there is a WPT after end that contains
the turn data for the next WPT.

For truncated routes with protoVer >= 0x07 a number of WPT,TDE pairs follow 
after the end instead of the single WPT. 

After the SPT an origo+scale combination is output. For protoVer >= 0x07 
a left\_right\_traffic may be output after the scale. If it is missing 
right is assumed.

After this preamble comes the first nav\_route\_point.

\subsection{Route datum structure layout}

The route datum structure contains two elements,
the action element which describes the function
of this datum, and the relevant structure for this
type of datum.

\begin{verbatim}

typedef struct nav_route_datum {
  uint16 action;    /* Type of structure. */
  union {
    struct nav_route_point              rp;
    struct nav_route_meta_point_origo   ro;
    struct nav_route_meta_point_scale   rs;
    struct nav_route_point_mini_delta_points rd;
    struct nav_route_point_micro_delta_points rdm;
    struct nav_route_point_meta         rm;
  } u;
} nav_route_datum_t;
typedef nav_route_datum_t* nav_route_datum_p;

\end{verbatim}



\subsection{Action field}

The action field determines which kind of route datum for the
current structure.

The action field is further subdivided by the following:

\begin{itemize}
\item Point Descriptions
\item Meta Data structures
\item Reserved values
\end{itemize}

\subsubsection{Point Descriptions}


\begin{verbatim}
  16             0
  0cceeexxxxxxxxxx
\end{verbatim}

The lower bits describe the action to be performed at this point.

The current values are defined at the moment.
\begin{itemize}
\item  0x0000 - {\tt nav\_route\_point\_end}
\item  0x0001 - {\tt nav\_route\_point\_start}
\item  0x0002 - {\tt  nav\_route\_point\_ahead     }
\item  0x0003 - {\tt  nav\_route\_point\_left      }
\item  0x0004 - {\tt  nav\_route\_point\_right     }
\item  0x0005 - {\tt  nav\_route\_point\_uturn     }
\item  0x0006 - {\tt  nav\_route\_point\_startat   }
\item  0x0007 - {\tt  nav\_route\_point\_finally   }
\item  0x0008 - {\tt  nav\_route\_point\_enter\_rdbt}
\item  0x0009 - {\tt  nav\_route\_point\_exit\_rdbt }
\item  0x000a - {\tt  nav\_route\_point\_ahead\_rdbt}
\item  0x000b - {\tt  nav\_route\_point\_left\_rdbt }
\item  0x000c - {\tt  nav\_route\_point\_right\_rdbt}
\item  0x000d - {\tt  nav\_route\_point\_exitat    }
\item  0x000e - {\tt  nav\_route\_point\_on        }
\item  0x000f - {\tt  nav\_route\_point\_park\_car  }
\item  0x0010 - {\tt  nav\_route\_point\_keep\_left }
\item  0x0011 - {\tt  nav\_route\_point\_keep\_right}
\item  0x0012 - {\tt  nav\_route\_point\_start\_with\_uturn}
\item  0x0013 - {\tt  nav\_rotue\_point\_uturn\_rdbt}
\item  0x0014 - {\tt  nav\_route\_point\_follow\_road}
\item  0x0015 - {\tt  nav\_route\_point\_enter\_ferry}
\item  0x0016 - {\tt  nav\_route\_point\_exit\_ferry}
\item  0x0017 - {\tt  nav\_route\_point\_change\_ferry}
\item  0x0018 - {\tt  nav\_route\_point\_endofroad\_left}
\item  0x0019 - {\tt  nav\_route\_point\_endofroad\_right}
\item  0x001A - {\tt  nav\_route\_point\_off\_ramp\_left}
\item  0x001B - {\tt  nav\_route\_point\_off\_ramp\_right}
\item  0x001C - {\tt  nav\_route\_point\_exit\_rdbt\_8}
\item  0x001D - {\tt  nav\_route\_point\_exit\_rdbt\_16}
\item  0x03fe - {\tt  nav\_route\_point\_delta      }
\item  0x03ff - {\tt  nav\_route\_point\_max       }

\end{itemize}

All these values mean that the rest of the structure
is filled as a nav\_route\_point.

The point description also includes information about the
crossing type for this point, if any.

The following values could be needed for crossing type:

\begin{itemize}

\item UNDEFINED\_CROSSING
\item NO\_CROSSING
\item CROSSING\_3WAYS\_T
\item CROSSING\_3WAYS\_Y
\item CROSSING\_4WAYS
\item CROSSING\_4WAYS\_ASYMMETRIC
\item CROSSING\_4WAYS\_X
\item CROSSING\_5WAYS
\item CROSSING\_6WAYS
\item CROSSING\_7WAYS
\item CROSSING\_8WAYS
\item CROSSING\_9WAYS
\item CROSSING\_10WAYS
\item CROSSING\_2ROUNDABOUT
\item CROSSING\_3ROUNDABOUT
\item CROSSING\_4ROUNDABOUT
\item CROSSING\_4ROUNDABOUT\_ASYMMETRIC
\item CROSSING\_5ROUNDABOUT
\item CROSSING\_6ROUNDABOUT
\item CROSSING\_7ROUNDABOUT

\end{itemize}

However, this would take too much space in the action member
to describe, so we instead define the following:

The most often occurring crossing types are defined in the
action member, and the rest can be described with another meta
structure.

These crossing types are defined in the action member.

\begin{itemize}
\item NO\_CROSSING
\item CROSSING\_3WAYS
\item CROSSING\_4WAYS
\item CROSSING\_MULTIWAY
\end{itemize}

This means that we need to use 2 bits for crossing type.

The middle bits, indicated with {\tt eee} are used for exit count.
The exitcount is used for roundabouts and turndescriptions to
allow commands such as ''Drive 530 meters then turn left
into the 5:th street S:t Petri Kyrkog''.

This means that we can describe up to 7 exits if the value zero
is reserved for ''unknown''.

There are two special turns {\tt  nav\_route\_point\_exit\_rdbt\_8} and
{\tt  nav\_route\_point\_exit\_rdbt\_16} they adds 8 and 16 to the exitcount.

\subsubsection{Meta data structure}

\begin{verbatim}
  16             0
  100000000zzzzzzz
\end{verbatim}

This gives us 127 possible metastructures.

\begin{itemize}

 \item Origo 0x8000
    \begin{verbatim}
    1000000000000000
    \end{verbatim}

 \item Scale 0x8001
    \begin{verbatim}
    1000000000000001
    \end{verbatim}

 \item Compressed trackpoints 0x8002-0x8003
    \begin{verbatim}
    1000000000000010   - Mini 2 x 2 x int16 plus speed limits
    1000000000000011   - Micro 5 x 2 x int8 (delta coded - currently unused).
    \end{verbatim}

 \item Reserved 0x8004

 \item Meta point 0x8005
    \begin{verbatim}
    1000000000000101   - Meta point, uint16 type + 4 x uint16
    \end{verbatim}


 \item TimeDistLeft 0x8006
    \begin{verbatim}
    1000000000000110   - Time, uint32,  and distance left, uint32
    \end{verbatim}
 
Tell the total time and distance left from the preseding trackpoint point.
The data is, after the uint16 type is a reserved uint16, a uint32 time left, ETG, 
and then an uint32 distance left to goal.

This packet is output for protoVer >= 0x07. It {\em must} be output 
after all nav\_route\_points and after some nav\_route\_point\_tracks.
Additionally it is output whenever it has been a long time since it was
last output.

 \item Landmark 0x8007
    \begin{verbatim}
    1000000000000111   - Text, type and distance.
    \end{verbatim}
    See \ref{Landmark}.

\item Lane Info 0x8008
    \begin{verbatim}
    1000000000001000   - Start of lanes group, or stop showing lanes.
    \end{verbatim}
    See \ref{Lane Info}.

\item Lane Data 0x8009
    \begin{verbatim}
    1000000000001001   - Several lanes in this, is after a Lane Info.
    \end{verbatim}
    See \ref{Lane Data}.

\item Signpost 0x800a
    \begin{verbatim}
    1000000000001010   - Text, background, text and front color.
    \end{verbatim}
    See \ref{Signpost}.
 
\end{itemize}

\subsubsection{Reserved values}

\begin{verbatim}
  16             0
  1wwwwwwwxxxxxxxx
\end{verbatim}

Where one of w is non-zero.

\subsection{nav\_route\_point}

This structure includes the following values:

\begin{verbatim}
  uint8 flags; /*   General flags. */
  uint8 speed_limit; /* Speedlimit to use for this segment. */
               /* The following speed limits have special values: */
               /* 0   -- No speed restriction (not in map or autobahn) */
               /* 254 -- No throughfare. */
               /* 255 -- No entry. */

  int16 x;     /* X position in 1 meters from local origo. */
  int16 y;     /* Y position in 1 meters from local origo. */
               /* We restrict ourselves to simple routes */
               /* of maximum 28 x 28 km ( +/- 14 km) before we need a */
               /* new origo and scale. */

  uint16 meters; /* Meters left to the next full trackpoint. I.e. */
                 /* distance left to the next waypoint or full */
                 /* trackpoint. Unused for protoVer >= 0x07 since */
                 /* the introduction of the TDE datum. */

  uint16 street_name_index; /* Index into the street name file. */
               /* The streetname index is an offset from a known */
               /* memory position to the correct string. */
               /* The strings are zero terminated and may */
               /* be divided by an 0xf7 character. */
               /* The text before the 0xf7 character is taken as */
               /* street name, and the text after as signpost text. */
\end{verbatim}

The flag field can be used to store additional data for the point. It 
is unused for protoVer < 0x07.

For protover 0x07 the flags field contains.

\begin{centering}
\begin{tabular}{|l|c|l|}
\hline
Name  & Size     & Description \\
      & (bits)   &             \\\hline
      & 6        & Unused      \\\hline
isLeftTraffic     & 1        & 0 - False, 1 - True \\\hline
ControlledAccess  & 1        & 1 - True, 0 - False  \\\hline
\end{tabular}
\end{centering}

In route with protoVer >= 0x07 a nav\_route\_point that is not a 
nav\_route\_point\_track {\em must} 
be followed by a TDE packet. This eases the task of calculating distances
between route points that imply direction. In addition some 
nav\_route\_point\_track datums must also be followed by TDEs, see below
(\ref{navroutepointtrack}).

\subsubsection{Start - SPT}

The startpoint is the first datum in the route. Since it is never referenced 
as a waypoint most fields are unused and undefined. The flags field however 
is special. One flag is defined, the start-with-uturn flag (0x01). It
indicates that the first segment may be traversed in the wrong direction and
that the driver in that case must turn around as soon as possible.

\subsection{nav\_route\_point\_track}
\label{navroutepointtrack}

This is a subtype of the nav\_route\_point which is used as a full
trackpoint, i.e. the point contains almost the same information as
the waypoints, but the point will never be presented to the user
other than indirectly (such as used for drawing the route etc).

In routes with protoVer < 0x07 the street\_name\_index is not used 
in this structure wheras in routes with protoVer >= 0x07 
street\_name\_index is used as normal.

In routes with protoVer >= 0x07 a nav\_route\_point\_track that 
constitutes a change in speedlimit, flags or streetname must
be followed by a TDE datum.

  
\subsection{Origo}

  The route origo point describes a point which is to be
  used as origo when calculating the local coordinates
  from the GPS-data (WGS-84)

\begin{verbatim}
  uint16 next_origo;   /* Index of the next origo point. */

  int32 origo_x;
                       /* X offset for center of the "map" */
                       /* that is used to translate to local */
                       /* coordinates. */
  int32 origo_y;
                       /* Y offset for center of the "map" */
                       /* that is used to translate to local */
                       /* coordinates. */
\end{verbatim}

\subsection{Scale}

  The route scale datum contains the scale factor to
  be used in the conversion of the GPS-data (WGS-84)
  to local coordinates.

  Observe that the Latitude scale factor is constant
  where ever on the globe we are.

  Fixed point value, scale\_x1 is integer part and
  scale\_x2 describes the fraction in 1/65536 parts.
  That is, scale\_x2 is calculated by
  scale\_x2 = (scale - scale\_x1)*65536
  Since (scale - scale\_x1) is between 1 and zero,
  scale\_x2 will be between 65536 and zero.


\begin{verbatim}
  int16 ref_x;            /* Reference to the last trackpoint. */
  int16 ref_y;            /* Reference to the last trackpoint. */

  uint32 scale_x1;        /* Scale factor in Longitude. */
                          /* Least significant 16 bits. */
  uint16 scale_x2;        /* Scale factor in Longitude. */
                          /* Most significant 32 bits. */
\end{verbatim}

  The ref\_* variables can be used as the coordinates of the previous
  route point in the new minimap.
  The ref\_* variables are defined as meters from the origo $ (0,0) $.

\subsection{Meta}

  The meta point can be used for any kind of values
  which need data in 16-bit units.

\begin{verbatim}
  uint16 type;
  uint16 a;
  uint16 b;
  uint16 c;
  uint16 d;
\end{verbatim}

The following types are defined:

\begin{itemize}

\item 0x1 Additional Text Information


\item 0x2 Additional Sound Information

Additional sound to be played at this point
More detailed sound information, Half way reached, Commercial etc.

\item 0x3 Additional Trackpoint Information

This is not used. DrivingOnRightSide is in attribute flags in WPT/TPT.
%Additional information needed for next trackpoint. Currently the
%first uint16 holds an extended flags section. Only two extended 
%flag is defined to describe left-hand/right-hand side driving. The 
%two least significant bits are 0=no change, 1=change to left-hand,
%2=change to right-hand driving and 3=unused.

\item 0x4 Segmented Route And End Is Near

Advise that new phonecall is made to Navigator Server to download
next part of a segmented route.

If a route is too large to transfer to the Navigator in one piece,
the route can be sent in a number of smaller segments with automatic
routing at the end of the segment.

\item 0x5 Report Point Reached

Instructs the Navigator to initiate contact with server to
report when the point is passed.

Can be used for vehicle tracking, automatic road pricing etc.

\end{itemize}

\subsection{Compressed trackpoints}

Full trackpoints are very often wasteful since only the position
and possibly speed values need to be updated. Therefore, we define 
{\em compressed trackpoints}
as multiple points that are packed into one structure.

There is two possible ways to pack multiple values into a structure:

\subsubsection{Micro points}

  This can be used to pack five trackpoints
  into a one compressed structure. The x and y offsets
  are in units from the last trackpoint, which can be any
  of a waypoint, a full trackpoint, a mini trackpoint or 
  another micro trackpoint.

  The last speed is given by a full or mini trackpoint is used.
  NB This may be suboptimal, it may be more efficient to keep
  all speed changes in full trackpoints and not in mini trackpoints.
  That way we do not have to scan for more than full trackpoints.

  Using 1 meters as the unit, we can describe points
  that lie within a square of 127 meters from the
  last point. Note that we must not accumulate a position outside
  of what is allowed in the minimap specified by origo and scale.

\begin{verbatim}

typedef struct nav_micro_delta_point {
  int8 x;
  int8 y;
} nav_micro_delta_point_t;
typedef nav_micro_delta_point_t *nav_micro_delta_point_p;

typedef struct nav_route_point_micro_delta_points {
  nav_micro_delta_point_t p1;
  nav_micro_delta_point_t p2;
  nav_micro_delta_point_t p3;
  nav_micro_delta_point_t p4;
  nav_micro_delta_point_t p5;
} nav_route_point_micro_delta_points_t;
typedef nav_route_point_micro_delta_points_t
    *nav_route_point_micro_delta_points_p;

\end{verbatim}

By definition, a nav\_route\_point\_micro\_track\_points\_t should
atleast contain one delta point and preferably not less than three.
(Since we otherwise could have used the
nav\_route\_point\_mini\_delta\_points\_t)

A point where both X and Y components are zero is invalid and
considered unused. All nav\_micro\_track\_point\_t after the
invalid one is also considered invalid.

NB. Micro track points are not implemented fully yet. {\em Do not use.}

\subsubsection{Mini points}

  X and Y offsets are the same as the one in nav\_route\_point\_track 
  (\ref{navroutepointtrack}).

\begin{verbatim}
typedef struct nav_mini_track_point {
  int16 x;
  int16 y;
} nav_mini_track_point_t;
typedef nav_mini_track_point_t *nav_mini_track_point_p;

typedef struct nav_route_point_mini_track_points {
  nav_mini_track_point_t p1;
  nav_mini_track_point_t p2;
  uint8 speed_limit1;
  uint8 speed_limit2;
} nav_route_point_mini_track_points_t;
typedef nav_route_point_mini_track_points_t
    *nav_route_point_mini_track_points_p;

\end{verbatim}

A nav\_route\_point\_mini\_track\_points\_t always
contains two track points. 

\subsection{Landmark}
\label{Landmark}

This datum may be appended once or several times after a 
wpt+tde combo. The wpt refered to is the one with the coordinate
of the first segment.

All extended landmarks that are active after 
a minimap change must be re-added after the Scale with 
start=stop=0. This allows a quick scanning of minimaps
without having to pay attention to the active landmarks.

Additional text information to be displayed between that point
in the route and the next crossing (the one the waypoint describes).

\begin{verbatim}
  uint16 flags;        /* Bit-mapped flags */

  uint16 street_name_index; /* Index into the street name file. */
                       /* The streetname index is an offset from a known */
                       /* memory position to the correct string. */
                       /* The strings are zero terminated. */
  uint16 id_and_startstop;
                       /* The least significant 14 bits are an id number
                          used to group starts/stops. 
                          The two most significant bits (0xc000) are the 
                          startstop bits. See below.
                              start    stop     meaning
                                0        0      Reminder of an extended landmark 
                                                placed after a Scale
                                1        0      Beginning of an extended landmark
                                0        1      End of an extended landmark
                                1        1      Point-like landmark
  int32 distance;
                       /* Distance from the landmark to 
                          the next crossing. */
\end{verbatim}

An extended landmark may have two data between the same 
waypoints if it both begins and ends between them. The
landmark datum with the start bit must come before the
one with the stop bit.

The flags field is bit coded as follows:

\begin{centering}
\begin{tabular}{|l|c|p{7cm}|}
\hline
Name  & Size     & Description \\
      & (bits)   &             \\\hline
Type  & 3        & 0 - Disturbed route, 1 - Landmark, 2-7 Reserved. \\\hline
Side  & 3        & 0 - Left, 1 - Right, 2 - Undefined. \\\hline
Landmark\_t &  5 & 0 - builtUpAreaLM \\
      &          & 1 - railwayLM \\
      &          & 2 - areaLM \\
      &          & 3 - poiLM \\
      &          & 4 - signPostLM \\
      &          & 5 - countryLM \\
      &          & 6 - countryAndBuiltUpAreaLM \\
      &          & 7 -  passedStreetLM \\
      &          & 8 -  accidentLM \\
      &          & 9 - roadWorkLM \\
      &          & 10 - cameraLM \\
      &          & 11 - speedTrapLM \\
      &          & 12 - policeLM \\
      &          & 13 - weatherLM \\
      &          & 14 - trafficOtherLM \\
      &          & 15 - userDefinedCameraLM \\
      &          & 16-31 - Reserved \\\hline
Landmark\_location\_t & 5 & 0 - after Indicates that the crossing is after the
                                landmark (you  should turn after the landmark)\\
      &          & 1 - before Indicates that the crossing is before the 
                              landmark.\\
      &          & 2 - in Indicates that the crossing is in the landmark 
                          (e.g. a park).\\
      &          & 3 - at Indicates that the crossing is at the landmark (if 
                          after, before or in does not fit the location).\\
      &          & 4 - pass Indicates that you should pass the landmark (when 
                            you drive ahead in a crossing or when the landmark 
                            is located just beside a road).\\
      &          & 5 - into Indicates that the crossing is in the landmark, and
                            you want description "after xx meters go into..". 
                            The landmark culd be e.g. a built-up area or a
                            country.\\
      &          & 6 - arrive  Indicates that the crossing is where you arrive 
                               at the landmark (from e.g. ferryitem).\\
      &          & 7 - 31 Reserved \\\hline
\end{tabular}
\end{centering}


\subsection{Lane Info}
\label{Lane Info}

This datum may be appended once or several times after a after a 
wpt+tde combo, after any landmarks. The wpt refered to is the one 
with the coordinate of the first segment.

The Lane Info may be followed by any number of Lane Data.
The Lane Infos are order in distance order with highest distance first.

If a Lane info is active after a mimimap change it must be re-added after the
Scale with Reminder of lanes = 1 and with all the Lane datas too.

The Lane Info contains the following.
\begin{verbatim}
  uint8 flags;        /* Bit-mapped flags */

  uint8 nbrLanes;     /* The number of lanes. */
  uint8 lane;         /* A lane. */
  uint8 lane;         /* A lane. */
  uint8 lane;         /* A lane. */
  uint8 lane;         /* A lane. */

  int32 distance;  /* Distance from the lanes to 
                      the next crossing. */
\end{verbatim}

\ \\
The flags field is bit coded as follows:

\begin{centering}
\begin{tabular}{|l|c|p{7cm}|}
\hline
Name  & Size     & Description \\
      & (bits)   &             \\\hline
Stop of lanes    & 1         & 0 - There be lanes here, 1 - Turn of Lanes until new Lane Info. \\\hline
Reminder of lanes    & 1         & 0 - Not a reminder, 1 - Reminder of lanes. \\\hline
      &          & Bit 2 - 15 Reserved \\\hline
\end{tabular}
\end{centering}

\ \\
\ \\
The lane uint8s contains the following. 
The lanes are ordered from left to right regardless of DrivingOnRightSide.
The lane's uint8 bit 7 is preferred lane or not and bit 6 is not allowed to
drive in with current vehicle and finally the 0-5 bits form a index into this
table.
%   1 - sharp left (135 degrees)
%  7 - sharp right (135 degrees)
%  8 - U-Turn Left (180 degrees)
%  9 - U-Turn Right (180 degrees)

\begin{verbatim}
  // Lane direction
  0 - no direction indicated
  1 - Reserved
  2 - left (90 degrees)
  3 - half left (45 degrees)
  4 - ahead  
  5 - half right (45 degrees)
  6 - right (90 degrees)
  7 - Reserved
  8 - Reserved
  9 - Reserved
 10-31 - Reserved
 32-63 - Reserved for client use (bit 5)
\end{verbatim}

\subsection{Lane Data}
\label{Lane Data}

This datum may be appended once or several times after a 
wpt+tde combo, after any landmarks and a lane info. The wpt referred to 
is the one with the coordinate of the first segment.
A Lane Data has 10 uint8s, where each uint8 represents a lane.

See Section \ref{Lane Info} for content of lane uint8s.


\subsection{Signpost}
\label{Signpost}

This datum may be appended once or several times after a 
wpt+tde combo, after any landmarks and lanes. The wpt referred to is the one 
with the coordinate of the first segment.
A Signpost is valid until the next wpt. The Signposts are in importance order
so the first one is more important than the next one.
If a mimimap change occurs any Signpost must be re-added after the
Scale.

The Signpost contains the following: Text, background, text and front
color (byte color index), dist. 

\begin{verbatim}
  uint16 signpost_text_index; /* Index into the string table. */

  byte textColor; /* The color of the text in the signpost. */
  byte backgroundColor; /* The color of the background of the signpost. */
  byte frontColor; /* The color of the front of the signpost. */

  byte reserved; /* Not used right now. */

  int32 distance;  /* Distance from the signpost to 
                      the next crossing. */
\end{verbatim}

The colors are indexes into the ngp param Route Color Table in the 
route reply.

\end{document}

%                latex2html -split 4 \
%                           -link 1 \
%                           -toc_depth 8 \
%                           -mkdir \
%                           -dir route_spec \
%                           -image_type gif \
%                           -antialias \
%                           -show_section_numbers \
%                           -auto_navigation \
%                           -index_in_navigation \
%                           -nocontents_in_navigation \
%                           -next_page_in_navigation \
%                           -previous_page_in_navigation \
%                           -local_icons \
%                           -transparent \
%                           -title "Route Specification" \
%                           -html_version 4.0,table,math \
%                           route.tex
