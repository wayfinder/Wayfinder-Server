% 
% 
% Copyright (c) 1999 - 2010, Vodafone Group Services Ltd
% All rights reserved.
% 
% Redistribution and use in source and binary forms, with or without modification, are permitted provided that the following conditions are met:
% 
%     * Redistributions of source code must retain the above copyright notice, this list of conditions and the following disclaimer.
%     * Redistributions in binary form must reproduce the above copyright notice, this list of conditions and the following disclaimer in the documentation and/or other materials provided with the distribution.
%     * Neither the name of the Vodafone Group Services Ltd nor the names of its contributors may be used to endorse or promote products derived from this software without specific prior written permission.
% 
% THIS SOFTWARE IS PROVIDED BY THE COPYRIGHT HOLDERS AND CONTRIBUTORS "AS IS" AND ANY EXPRESS OR IMPLIED WARRANTIES, INCLUDING, BUT NOT LIMITED TO, THE IMPLIED WARRANTIES OF MERCHANTABILITY AND FITNESS FOR A PARTICULAR PURPOSE ARE DISCLAIMED. IN NO EVENT SHALL THE COPYRIGHT HOLDER OR CONTRIBUTORS BE LIABLE FOR ANY DIRECT, INDIRECT, INCIDENTAL, SPECIAL, EXEMPLARY, OR CONSEQUENTIAL DAMAGES (INCLUDING, BUT NOT LIMITED TO, PROCUREMENT OF SUBSTITUTE GOODS OR SERVICES; LOSS OF USE, DATA, OR PROFITS; OR BUSINESS INTERRUPTION) HOWEVER CAUSED AND ON ANY THEORY OF LIABILITY, WHETHER IN CONTRACT, STRICT LIABILITY, OR TORT (INCLUDING NEGLIGENCE OR OTHERWISE) ARISING IN ANY WAY OUT OF THE USE OF THIS SOFTWARE, EVEN IF ADVISED OF THE POSSIBILITY OF SUCH DAMAGE.
%
%  

\section{Summary}
The main purpose of this interface is to provide a general API to
the \mc-system. The API can be used by a third party to incorporate
our services into their system. The main service for the \mc-system 
is to provide an answer to the question ``How do I best go from A to B?''. 
The origin and destination can for example be given to the system in terms of a 
coordinate, a street address, or a company name. Multiple destinations 
will result in a route from the origin to the closest destination.

The following services are available:
\begin{description}
   \item[Search] By sending a text search string into the \mc-system the matching
      items are returned to the caller. These items could then be used as
      origins or destinations. If the answer contains a category, this can
      be expanded to its content. Some optional parameters can be provided
      to make it possible for the user to set the search criterion.\\
      \textbf{Example:} By sending a request containing ``\texttt{Lund,
         Bar}'' the answer might contain the street item ``\texttt{Barav�gen}'' and 
         the company item ``\texttt{Bara Elektronik AB}'', both located in the
         city of Lund.
   \item[Route] It is possible to send an origin and a destination in a route request to
      the \mc-system and get
      the best route between them in reply. The origin and destinations
      can be specified either as coordinates, items from a previous search
      requests, or an entire
      category of companies or objects. Some optional
      parameters can be provided to 
      adjust the choice of the optimal route.\\
      \textbf{Example 1:} A request that contains the coordinate
         \texttt{(55.718, 13.190)} as origin and the point of interest
         named ``\texttt{Lund, Station}'' 
         as destination, will get a reply containing a route from the street address
         that is closest to the given coordinate, (Barav�gen 1), to the
         Central Train Station of Lund.\\
      \textbf{Example 2:} If the request contains the company item ``\texttt{Lund, 
         Wayfinder Systems}'' as origin, and both the restaurant
         ``\texttt{Lund, Pizzeria Portofino}'' and the restaurant
         ``\texttt{Lund, Pizzeria F�ladstorget}'' as 
         destinations, the reply will contain a route from Wayfinder 
         Systems to the closest one of the two pizzerias.\\
      \textbf{Example 3:} A request, that contains the company item
         ``\texttt{Lund, Wayfinder Systems}'' as origin and the company
         category item ``\texttt{Restaurant}'' as destination, will be replied to with 
         a route to the restaurant that is closest to the address of
         Wayfinder Systems.
   \item[Expand] Items, including categories of companies, that can be returned
      as a reply to a Search request may be expanded. By expanding a category
      we mean that the
      items in that category will be returned.\\ % Better explanation
      \textbf{Example:} Sending an expand request for the area item ``\texttt{Lund}'' 
         and the category item ``\texttt{Restaurant}'', will result in a reply containing a 
         list of all restaurants in Lund.
   \item[User handling] It is possible to update the user 
      profiles by sending a user request to the \mc-system. The user
      profiles contains information about the user's vehicle, phone 
      and other preferences that the user might have. User handling
      functions also include user login, verification of sessions,
      user logout, logging and debiting.
   \item[Positioning] It is possible to ask for the approximate position of a
     certain mobile phone. The availability of this feature is
     currently limited
     by network hardware availibility, and user privacy
     considerations.
   \item[SMS] Route descriptions or other text may be formatted for
     display on an SMS capable phone, and the SMSes may be sent to the
     mobile phone of a user.
   \item[E-mail] Route descriptions or other text may be formatted and
     sent as e-mails.
   \item[Traffic] The caller can get graphic traffic information for a specified area. 
\end{description}
